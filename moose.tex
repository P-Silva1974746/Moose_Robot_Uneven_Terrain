\documentclass[11pt,twocolumn]{article}

\usepackage{amsmath}
\usepackage{graphicx}
\usepackage{geometry}
\geometry{margin=0.75in}
\usepackage{titlesec}
\usepackage{caption}
\usepackage{tikz}
\usetikzlibrary{arrows.meta, positioning, shapes.geometric}
\usepackage{float}            
\usepackage{newtxtext,newtxmath} 
\usepackage{microtype}    
\usepackage{booktabs}
\usepackage{graphicx}
\usepackage{hyperref}        

\titleformat{\section}{\large\bfseries}{\thesection.}{1em}{}
\titleformat{\subsection}{\normalsize\bfseries}{\thesubsection.}{1em}{}
\captionsetup[figure]{font=small}

\title{Deep Reinforcement Learning for Efficient Navigation of the Moose Robot}
\author{Constança Fernandes \and Isabel Sá \and Pedro Silva}
\date{\today}


\begin{document}

\twocolumn[
\maketitle

\begin{abstract}
\noindent
This paper presents a deep reinforcement learning (DRL) approach for efficient navigation of the Moose robot in complex environments with unknown rough terrain. We train an Advantage Actor-Critic (A2C) model to optimise the robot's trajectory while maintaining navigation performance. The proposed solution is validated in simulation using diverse terrain maps, and its effectiveness is compared with traditional planning algorithms such as Greedy and A*.
\end{abstract}
\vspace{1em}
]

\section{Introduction}
Mobile robots face significant challenges when navigating rough terrain, particularly in terms of balancing robustness and speed. The Moose robot, equipped with various sensors, provides a suitable platform for exploring intelligent navigation strategies. The aim of this work is to train a DRL model that allows Moose to autonomously reach target locations efficiently, optimising its trajectory for reliable navigation.

\section{Related Work}

Several approaches have been proposed for mobile robot navigation, including traditional path planners such as A* and greedy heuristics. These classical algorithms often rely on complete and accurate terrain information and do not explicitly optimize trajectory quality in uncertain or uneven environments. More recently, deep reinforcement learning (DRL) has shown great promise in addressing complex control tasks for navigation in rough terrain.

Notably, Zhang et al. \cite{Zhang2018DRL} introduced a DRL-based framework for robot navigation over unknown rough terrain, demonstrating the potential of neural policy learning to adapt to complex and uncertain environments without explicit prior knowledge of the terrain. This approach enables autonomous agents to learn navigation policies that implicitly encode safety and efficiency considerations.

Subsequent works have incorporated terrain classification and slip estimation through deep learning methods to enhance robot perception and control \cite{gonzalez2018deepterramechanics}. Others have modeled terrain traversability and energy consumption of legged robots using inverse reinforcement learning, enabling more energy-efficient locomotion policies \cite{gan2022energy}.

Recent advances also include the development of dynamic slip cost mapping techniques to improve autonomous navigation safety and efficiency \cite{slipnet2024}. Additionally, domain randomization techniques have been evaluated to improve the generalization capabilities of DRL locomotion policies trained in simulation for real-world applications \cite{mdpi2022randomization}.

Reliable planning approaches combining terrain awareness and DRL have also been proposed to improve navigation robustness in uneven outdoor environments \cite{weerakoon2022terp}.

Our approach combines terrain-aware perception, risk-aware planning, and deep reinforcement learning to jointly optimize navigation safety and energy efficiency in uneven outdoor environments, addressing limitations found in prior works by integrating these components into a unified framework.


\section{Methodological Approach}
In this section, we describe the experimental setup and the navigation strategies evaluated in our study. The Moose robot operates in a simulated uneven terrain environment, where each episode consists of a navigation task from a start point to a goal point. To ensure a fair and generalizable evaluation, the start and goal positions are randomly assigned at the beginning of every episode, for all tested methods—including Greedy, A*, and A2C-based reinforcement learning. This randomized configuration prevents overfitting to specific trajectories and promotes the development of robust navigation policies.

\subsection{Terrain Generation Using Perlin Noise}

To support the evaluation of navigation strategies in diverse topographical conditions, terrain surfaces are procedurally synthesised using Perlin noise — a gradient-based method known for generating smooth and naturalistic elevation profiles. This approach allows for the creation of continuous, non-repetitive landscapes that emulate real-world geographical features such as slopes, hills, and valleys.

The simulation environment is built within Webots and incorporates a  200\(\times\)200 that defines a three-dimensional surface.Terrain generation is based on a custom implementation of 2D Perlin noise, configured with a single octave and no bias correction. The resulting noise values are scaled to improve terrain characteristics, producing realistic elevation differences suitable for evaluating the robot's mobility and navigation performance.

To integrate the generated elevation data, the world model is modified programmatically, replacing the standard height map with the synthesised terrain. The region immediately surrounding the robot's initial position is flattened during a post-processing step to ensure a stable start to the episode. The robot's initial coordinates are then adjusted to match this modified elevation, ensuring consistency between the terrain and the spawn location.

The procedural generation pipeline includes tools for analysing and visualising terrain characteristics. Statistical summaries — such as mean, standard deviation, and elevation range — are calculated to assess the spatial variability of each generated map. In addition, elevation surface heatmaps are produced using perceptually uniform colour gradients, allowing for quick visual inspection of terrain diversity.

\begin{figure}[htbp]
    \centering
    \includegraphics[width=\columnwidth]{heatmap_terrain.png}
    \caption{Heatmap visualization of the generated terrain elevation using Perlin noise. The colour gradient reflects elevation levels, with warmer colours indicating higher elevations.}
    \label{fig:heatmap_terrain}
\end{figure}


This synthetic terrain generation framework supports the randomisation of domains important for training, allowing learning and planning algorithms to generalise to a wide variety of environmental conditions. By adjusting noise parameters such as scale and frequency, it is possible to control the complexity and slope of the generated maps, facilitating systematic testing of the robustness of navigation policies.

\subsection{Greedy Navigation Controller}

As a baseline control strategy, a rule-based greedy navigation controller was implemented to guide the robot towards the target using only local sensor data.

At each simulation step, the robot estimates its current heading and position relative to the goal using data from its GPS and IMU sensors. The heading angle is extracted from the yaw component of the IMU, while the displacement vector to the goal is derived from GPS coordinates.

The controller prioritises actions that reduce the Euclidean distance to the target:
\[
d_t = \sqrt{(x_g - x_t)^2 + (y_g - y_t)^2}
\]
It then calculates the angular deviation $\Delta\theta$ between the robot's orientation and the vector to the goal. If $|\Delta\theta| > 0.01745$ radians (1 degree), the robot rotates in place to align with the goal direction. Once alignment is achieved, it proceeds forward with a constant velocity.

In addition to heading control, the robot monitors its pitch and roll values continuously. If these exceed a predefined inclination threshold (indicating unsafe slopes), the robot prioritises stability by entering a recovery manoeuvre: it induces rotation in the direction opposite to the tilt. This behaviour prevents traversal of steep regions and reduces the risk of falling or flipping, thereby improving navigation safety.

\paragraph{Recovery Behaviour}

To detect stagnation or local minima, the controller compares successive GPS positions. If minimal displacement is observed over a defined window, the robot performs a rotational escape routine. This heuristic aims to dislodge the robot from potentially blocked or unstable terrain.

\paragraph{Termination Criteria}

An episode terminates under the following conditions:

\begin{itemize}
    \item The robot reaches within a specified proximity to the goal.
    \item A timeout of 120 seconds is exceeded.
\end{itemize}

Despite its simplicity, this greedy strategy establishes a valuable performance baseline. It enables comparisons against planning-based and learning-based approaches under diverse terrain conditions.

\subsection{Path Planning with A* Algorithm}

Navigation commences with a global planning module based on the A* search algorithm, adapted to account for terrain irregularities. The operational environment is discretised into a two-dimensional grid, with each cell assigned a traversal cost influenced by local terrain features, such as slope and elevation.

Each cell considers an \textbf{8-connected neighborhood}, allowing transitions in both axial and diagonal directions. Diagonal movements incur higher traversal costs, reflecting their increased Euclidean distance ($\sqrt{2}$) relative to axial steps. This preserves geometric consistency and penalises sharp turns on rugged terrain.

The pathfinding cost function is defined as:
\[
f(n) = g(n) + h(n)
\]
where $g(n)$ represents the cumulative cost from the origin to node $n$, computed as the Euclidean distance scaled by a terrain roughness factor derived from local elevation gradients. 

To ensure physically plausible trajectories, impassable areas and steep inclines are assigned prohibitively high traversal costs. The slope at each cell is estimated via numerical gradients over a local elevation window, and terrain flatness is evaluated by measuring the terrain gradient derived from local elevation differences.

The heuristic term $h(n)$ estimates the remaining cost to the goal:
\[
h(n) = \lambda \cdot \sqrt{(x_g - x_n)^2 + (y_g - y_n)^2}
\]
where $\lambda$ modulates the influence of distance based on estimated terrain roughness. This formulation guides the planner towards geometrically direct yet traversable paths that respect the Moose’s locomotion constraints.

This enhanced cost model enables the planner to favour routes that avoid steep or unstable regions, promoting both efficiency and safety in terrain-constrained environments.
\subsection{Robot Control via A2C}

An alternative control architecture employs the Advantage Actor-Critic (A2C) algorithm. In this setup, a reinforcement learning (RL) agent interacts with the robot via a TCP/IP socket interface, enabling external control of motor velocities based on continuous feedback from the environment.

\paragraph{Action Space}

The robot operates in a continuous action space:
\[
a = [v_{\text{left}}, v_{\text{right}}], \quad v \in [-1, 1]
\]
where $v_{\text{left}}$ and $v_{\text{right}}$ represent the normalized velocities of the left and right wheels, respectively.

\paragraph{Observation Space}

The agent receives a six-dimensional observation vector consisting of:

\begin{itemize}
    \item 3D position $(x, y, z)$ obtained via GPS.
    \item Orientation $(\phi, \theta, \psi)$ (roll, pitch, yaw) obtained from the IMU.
\end{itemize}

This state representation provides sufficient spatial awareness for navigating uneven terrain.

\paragraph{Reward Function}

The reward signal \( R_t \) at time step \( t \) is defined as:
\[
R_t = 
\begin{cases}
0.5 + \frac{d_{\text{start}}}{T_{\text{total}}} & \text{if goal reached} \\
\mu \cdot \frac{d_{\min} - d_t}{\Delta t} & \text{if progress is made} \\
-0.5 & \text{if stuck too long} \\
0 & \text{otherwise}
\end{cases}
\]
where \( d_t \) is the current distance to the goal, \( d_{\min} \) is the minimal distance reached so far, \( T_{\text{total}} \) is the elapsed time, and \( \mu \) is a scaling factor.

This reward function follows the design proposed by Zhang et al. \cite{Zhang2018DRL}.

\subsection{Reward Structure and Termination Conditions}

The reward function is designed to promote efficient and goal-directed navigation:

\begin{itemize}
    \item \textbf{Success:} Bonus proportional to initial distance and time efficiency.
    \item \textbf{Progress:} Reward for reducing distance to the goal.
    \item \textbf{Stagnation:} Penalty for extended inactivity.
    \item \textbf{Timeout:} Episode terminates with penalty after 120 seconds without reaching the goal.
\end{itemize}

Stuck conditions are detected by monitoring positional changes over time, preventing indefinite non-progress states.

\subsection{Reinforcement Learning Environment}

The control architecture is implemented as a Gymnasium-compatible environment, \texttt{MooseEnv}, which abstracts the Webots simulation and facilitates interaction with standard RL algorithms. It defines:

\begin{itemize}
    \item A continuous action space for wheel velocities in \([-1, 1]\).
    \item A continuous observation space with 3D position and orientation.
    \item \texttt{reset} and \texttt{step} methods that manage simulation episodes via socket-based communication.
\end{itemize}

This modular interface enables seamless experimentation with various RL algorithms, decoupling control policies from simulation logic.

\subsection{Reinforcement Learning Training Setup}

Training is conducted using the \texttt{Stable Baselines3} implementation of A2C. The script \texttt{train.py} automates the training workflow:

\begin{itemize}
    \item Initializes the \texttt{MooseEnv} environment wrapping the Webots simulation.
    \item Creates the directory \texttt{models/a2c\_moose} for storing model checkpoints.
    \item If a pretrained model (\texttt{a2c\_moose\_model.zip}) is found, it resumes training from that checkpoint.
    \item Otherwise, it initializes a new A2C agent with an MLP policy, learning rate of $1 \times 10^{-4}$, and TensorBoard logging enabled.
    \item The agent is trained for 100,000 timesteps with verbose logging.
    \item Upon completion, the model is saved to disk.
\end{itemize}

\section{Experimental Evaluation}

This section presents the evaluation results of the navigation algorithms Greedy and A* over 100 runs (10 runs on each of 10 different terrain maps). The performance metrics considered include planned distance, actual distance traveled, completion time, average speed, and navigation outcome (success or failure).

\subsection{Greedy Algorithm}

Table~\ref{tab:greedy_summary} shows the aggregated statistics for the Greedy algorithm across the 100 runs.

\begin{table}[ht]
\centering
\caption{Summary statistics for the Greedy algorithm (100 runs).}
\label{tab:greedy_summary}
\resizebox{\columnwidth}{!}{%
\begin{tabular}{lrrrrr}
\toprule
Metric  & Mean & Std Dev & Min & Median & Max \\
\midrule
Distance (m)  & 109.48 & 40.18 & 29.89 & 100.50 & 190.34 \\
Distance Traveled (m)  & 110.44 & 40.10 & 22.09 & 103.63 & 190.55 \\
Time (s)  & 117.41 & 43.64 & 20.03 & 105.15 & 211.72 \\
Average Speed (m/s)  & 0.98 & 0.21 & 0.57 & 0.97 & 1.53 \\
Result (Success=1)  & 0.80 & 0.40 & 0 & 1 & 1 \\
\bottomrule
\end{tabular}
}
\end{table}

The Greedy algorithm achieved a success rate of 80\%, with an average completion time of 117.41 seconds. The average speed was 0.98 m/s. The small difference between planned and traveled distance indicates generally efficient path following, although some variability is present.

\subsection{A* Algorithm}

Table~\ref{tab:astar_summary} presents the summary statistics for the A* algorithm over the same set of runs.

\begin{table}[ht]
\centering
\caption{Summary statistics for the A* algorithm (100 runs).}
\label{tab:astar_summary}
\resizebox{\columnwidth}{!}{%
\begin{tabular}{lrrrrr}
\toprule
Metric  & Mean & Std Dev & Min & Median & Max \\
\midrule
Distance (m)  & 112.02 & 37.11 & 52.23 & 103.33 & 179.67 \\
Distance Traveled (m)  & 113.67 & 37.56 & 52.23 & 103.34 & 183.99 \\
Time (s)  & 109.03 & 39.89 & 33.32 & 94.00 & 214.97 \\
Average Speed (m/s)  & 1.04 & 0.19 & 0.58 & 1.05 & 1.65 \\
Result (Success=1)  & 0.85 & 0.36 & 0 & 1 & 1 \\
\bottomrule
\end{tabular}
}
\end{table}

The A* algorithm demonstrated a higher success rate of 85\% and slightly better average speed (1.04 m/s) compared to Greedy. The average completion time was lower at 109.03 seconds, reflecting the efficiency of the path planning.

\subsection{Comparison and Discussion}

The comparative analysis indicates that the A* algorithm outperforms the Greedy method in terms of success rate, speed, and completion time. Although distances planned and traveled are comparable, the lower time and higher speed achieved by A* suggest more efficient navigation paths and better adaptation to terrain.


\section{Conclusions and Future Work}

This study demonstrates the feasibility of using deep reinforcement learning to improve efficiency in robot navigation. While DRL outperforms heuristic-based approaches in varied environments, future work should focus on transfer learning for real-world deployment and adapting the model for dynamic obstacles.

\bibliographystyle{plain}
\bibliography{references}

\end{document}
