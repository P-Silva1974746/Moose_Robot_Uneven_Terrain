\documentclass[11pt,twocolumn]{article}

\usepackage{amsmath}
\usepackage{graphicx}
\usepackage{geometry}
\geometry{margin=0.75in}
\usepackage{titlesec}
\usepackage{caption}
\usepackage{tikz}
\usetikzlibrary{arrows.meta, positioning, shapes.geometric}
\usepackage{float}            
\usepackage{newtxtext,newtxmath} 
\usepackage{microtype}    
\usepackage{booktabs}
\usepackage{hyperref}        

\titleformat{\section}{\large\bfseries}{\thesection.}{1em}{}
\titleformat{\subsection}{\normalsize\bfseries}{\thesubsection.}{1em}{}
\captionsetup[figure]{font=small}

\title{Deep Reinforcement Learning for Efficient Navigation of the Moose Robot}
\author{Constança Fernandes \and Isabel Sá \and Pedro Silva}
\date{\today}


\begin{document}

\twocolumn[
\maketitle

\begin{abstract}
\noindent
This paper presents a deep reinforcement learning (DRL) approach for efficient navigation of the Moose robot in complex environments with unknown rough terrain. We train an Advantage Actor-Critic (A2C) model to optimise the robot's trajectory while maintaining navigation performance. The proposed solution is validated in simulation using diverse terrain maps, and its effectiveness is compared with traditional planning algorithms such as Greedy and A*.
\end{abstract}
\vspace{1em}
]

\section{Introduction}
Mobile robots face significant challenges when navigating rough terrain, particularly in terms of balancing robustness and speed. The Moose robot, equipped with various sensors, provides a suitable platform for exploring intelligent navigation strategies. The aim of this work is to train a DRL model that allows Moose to autonomously reach target locations efficiently, optimising its trajectory for reliable navigation.

\section{Related Work}

Several approaches have been proposed for mobile robot navigation, including traditional path planners like A* and greedy heuristics. These algorithms often rely on complete terrain information and do not explicitly optimise trajectory quality in uncertain environments. More recently, deep reinforcement learning has shown promise in solving complex control problems. Notably, Zhang et al. \cite{Zhang2018DRL} introduced a DRL-based framework for rough terrain navigation, demonstrating the potential of neural policy learning approaches in this domain.
\section{Methodological Approach}

\subsection{Terrain Generation Using Perlin Noise}

To support the evaluation of navigation strategies in diverse topographical conditions, terrain surfaces are procedurally synthesised using Perlin noise — a gradient-based method known for generating smooth and naturalistic elevation profiles. This approach allows for the creation of continuous, non-repetitive landscapes that emulate real-world geographical features such as slopes, hills, and valleys.

The simulation environment is built within Webots and incorporates a  200\(\times\)200 at defines a three-dimensional surface. Terrain generation is based on a custom implementation of 2D Perlin noise, configured with a single octave and no bias correction. The resulting noise values are scaled to improve terrain characteristics, producing realistic elevation differences suitable for evaluating the robot's mobility and navigation performance.

To integrate the generated elevation data, the world model is modified programmatically, replacing the standard height map with the synthesised terrain. The region immediately surrounding the robot's initial position is flattened during a post-processing step to ensure a stable start to the episode. The robot's initial coordinates are then adjusted to match this modified elevation, ensuring consistency between the terrain and the spawn location.

The procedural generation pipeline includes tools for analysing and visualising terrain characteristics. Statistical summaries — such as mean, standard deviation, and elevation range — are calculated to assess the spatial variability of each generated map. In addition, elevation surface heatmaps are produced using perceptually uniform colour gradients, allowing for quick visual inspection of terrain diversity.

This synthetic terrain generation framework supports the randomisation of domains important for training, allowing learning and planning algorithms to generalise to a wide variety of environmental conditions. By adjusting noise parameters such as scale and frequency, it is possible to control the complexity and slope of the generated maps, facilitating systematic testing of the robustness of navigation policies.

\subsection{Path Planning with A* Algorithm}

Navigation commences with a global planning module based on the A* search algorithm, adapted to account for terrain irregularities. The operational environment is discretised into a two-dimensional grid, with each cell assigned a traversal cost influenced by local terrain features, such as slope and elevation.

The pathfinding cost function is defined as:
\[
f(n) = g(n) + h(n)
\]
where $g(n)$ represents the cumulative cost from the origin to node $n$, computed as the Euclidean distance scaled by a terrain roughness factor derived from local elevation gradients. 

To ensure physically plausible trajectories, impassable areas and steep inclines are assigned prohibitively high traversal costs. The slope at each cell is estimated via numerical gradients over a local elevation window, and terrain flatness is evaluated by checking the alignment of the local surface normal with the vertical axis.

The heuristic term $h(n)$ estimates the remaining cost to the goal:
\[
h(n) = \lambda \cdot \sqrt{(x_g - x_n)^2 + (y_g - y_n)^2}
\]
where $\lambda$ modulates the influence of distance based on estimated terrain roughness. This formulation guides the planner towards geometrically direct yet traversable paths that respect the robot’s locomotion constraints.

This enhanced cost model enables the planner to favour routes that avoid steep or unstable regions, promoting both efficiency and safety in terrain-constrained environments.

\subsection{Robot Control via A2C}

An alternative control architecture employs the Advantage Actor-Critic (A2C) algorithm. In this setup, a reinforcement learning (RL) agent interacts with the robot via a TCP/IP socket interface, enabling external control of motor velocities based on continuous feedback from the environment.

\paragraph{Action Space}

The robot operates in a continuous action space:
\[
a = [v_{\text{left}}, v_{\text{right}}], \quad v \in [-1, 1]
\]
where $v_{\text{left}}$ and $v_{\text{right}}$ represent the normalised velocities of the left and right wheels, respectively.

\paragraph{Observation Space}

The agent receives a six-dimensional observation vector consisting of:

\begin{itemize}
    \item 3D position $(x, y, z)$ obtained via GPS.
    \item Orientation $(\phi, \theta, \psi)$ (roll, pitch, yaw) obtained from the IMU.
\end{itemize}

This state representation provides sufficient spatial awareness for navigating uneven terrain.

\paragraph{Reward Function}

The reward signal \( R_t \) at time step \( t \) is defined as:
\[
R_t = 
\begin{cases}
0.5 + \frac{d_{\text{start}}}{T_{\text{total}}} & \text{if goal reached} \\
\mu \cdot \frac{d_{\min} - d_t}{\Delta t} & \text{if progress is made} \\
-0.5 & \text{if stuck too long} \\
0 & \text{otherwise}
\end{cases}
\]
where \( d_t \) is the current distance to the goal, \( d_{\min} \) is the minimal distance reached so far, \( T_{\text{total}} \) is the elapsed time, and \( \mu \) is a scaling factor.

\subsection{Reward Structure and Termination Conditions}

The reward function is designed to promote efficient and goal-directed navigation:

\begin{itemize}
    \item \textbf{Success:} Bonus proportional to initial distance and time efficiency.
    \item \textbf{Progress:} Reward for reducing distance to the goal.
    \item \textbf{Stagnation:} Penalty for extended inactivity.
    \item \textbf{Timeout:} Episode terminates with penalty after 120 seconds without reaching the goal.
\end{itemize}

Stuck conditions are detected by monitoring positional changes over time, preventing indefinite non-progress states.

\subsection{Reinforcement Learning Environment}

The control architecture is implemented as a Gymnasium-compatible environment, \texttt{MooseEnv}, which abstracts the Webots simulation and facilitates interaction with standard RL algorithms. It defines:

\begin{itemize}
    \item A continuous action space for wheel velocities in \([-1, 1]\).
    \item A continuous observation space with 3D position and orientation.
    \item \texttt{reset} and \texttt{step} methods that manage simulation episodes via socket-based communication.
\end{itemize}

This modular interface enables seamless experimentation with various RL algorithms, decoupling control policies from simulation logic.

\subsection{Reinforcement Learning Training Setup}

Training is conducted using the \texttt{Stable Baselines3} implementation of A2C. The script \texttt{train.py} automates the training workflow:

\begin{itemize}
    \item Initializes the \texttt{MooseEnv} environment wrapping the Webots simulation.
    \item Creates the directory \texttt{models/a2c\_moose} for storing model checkpoints.
    \item If a pretrained model (\texttt{a2c\_moose\_model.zip}) is found, it resumes training from that checkpoint.
    \item Otherwise, it initializes a new A2C agent with an MLP policy, learning rate of $1 \times 10^{-4}$, and TensorBoard logging enabled.
    \item The agent is trained for 100,000 timesteps with verbose logging.
    \item Upon completion, the model is saved to disk.
\end{itemize}


\section{Experimental Evaluation}

This section presents the evaluation results of the navigation algorithms Greedy and A* over 100 runs (10 runs on each of 10 different terrain maps). The performance metrics considered include planned distance, actual distance traveled, completion time, average speed, and navigation outcome (success or failure).

\subsection{Greedy Algorithm}

Table~\ref{tab:greedy_summary} shows the aggregated statistics for the Greedy algorithm across the 100 runs.

\begin{table}[H]
\centering
\caption{Summary statistics for the Greedy algorithm (100 runs).}
\label{tab:greedy_summary}
\resizebox{\columnwidth}{!}{%
\begin{tabular}{lrrrrrr}
\toprule
Metric  & Mean & Std Dev & Min & Median & Max \\
\midrule
Distance (m)  & 109.48 & 40.18 & 29.89 & 100.50 & 190.34 \\
Distance Traveled (m)  & 110.44 & 40.10 & 22.09 & 103.63 & 190.55 \\
Time (s)  & 117.41 & 43.64 & 20.03 & 105.15 & 211.72 \\
Average Speed (m/s)  & 0.98 & 0.21 & 0.57 & 0.97 & 1.53 \\
Result (Success=1)  & 0.80 & 0.40 & 0 & 1 & 1 \\
\bottomrule
\end{tabular}
\end{table}
}

The Greedy algorithm achieved a success rate of 80\%, with an average completion time of 117.41 seconds. The average speed was 0.98 m/s. The small difference between planned and traveled distance indicates generally efficient path following, although some variability is present.

\subsection{A* Algorithm}

Table~\ref{tab:astar_summary} presents the summary statistics for the A* algorithm over the same set of runs.

\begin{table}[H]
\centering
\caption{Summary statistics for the A* algorithm (100 runs).}
\label{tab:astar_summary}
\resizebox{\columnwidth}{!}{%
\begin{tabular}{lrrrrrr}
\toprule
Metric  & Mean & Std Dev & Min & Median & Max \\
\midrule
Distance (m)  & 112.02 & 37.11 & 52.23 & 103.33 & 179.67 \\
Distance Traveled (m)  & 113.67 & 37.56 & 52.23 & 103.34 & 183.99 \\
Time (s)  & 109.03 & 39.89 & 33.32 & 94.00 & 214.97 \\
Average Speed (m/s)  & 1.04 & 0.19 & 0.58 & 1.05 & 1.65 \\
Result (Success=1)  & 0.85 & 0.36 & 0 & 1 & 1 \\
\bottomrule
\end{tabular}
\end{table}
}

The A* algorithm demonstrated a higher success rate of 85\% and slightly better average speed (1.04 m/s) compared to Greedy. The average completion time was lower at 109.03 seconds, reflecting the efficiency of the path planning.

\subsection{Comparison and Discussion}

The comparative analysis indicates that the A* algorithm outperforms the Greedy method in terms of success rate, speed, and completion time. Although distances planned and traveled are comparable, the lower time and higher speed achieved by A* suggest more efficient navigation paths and better adaptation to terrain.


\section{Conclusions and Future Work}

This study demonstrates the feasibility of using deep reinforcement learning to improve efficiency in robot navigation. While DRL outperforms heuristic-based approaches in varied environments, future work should focus on transfer learning for real-world deployment and adapting the model for dynamic obstacles.

\bibliographystyle{plain}
\bibliography{references}

\end{document}
